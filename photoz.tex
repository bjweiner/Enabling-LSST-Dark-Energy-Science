% ======================================================================

\section{Photo-z}
\label{sec:photoz}

{\em A) Photo-z training and calibration (including in cluster regime,
spectroscopy of blends identified in space-based imaging).}

% For each science area, we should:
%
% 1) Describe the science need in a few sentences at most
%
% 2) Describe the needed capability (or capabilities) in enough detail
% that someone could determine whether they overlap with other needs.
% (E.g.:  “medium-resolution (R~4000-6000) spectroscopy covering the
% full optical window for i<25.3 objects with multiplexing of ~1000 over
% ~10 arcminute diameter fields” could work, or for a differentcase
% “optical medium-resolution spectroscopy with a highly-multiplexed
% spectrograph with a many-degree FoV on a 4m telescope” would also give
% enough information to allow identification of common needs).

% ----------------------------------------------------------------------

\subsection{Ambiguous Blending}
Approximately 14\% of the objects detected in the LSST survey will be ambiguous
blends of two or more galaxies (Dawson et al. 2016), that is galaxies with
separations $\lesssim$ the PSF width that fail to be identified as blended and
are detected as a single object. These ambiguous blends pose a challenge and
potential source of systematic for photometric redshift algorithms operating
under the assumption of a single isolated galaxy (currently all photo-z
algorithms). By definition these ambiguous blends will go undetected so it seems
that the best strategy is calibration. In this regard it seems like the best
strategy is to use overlapping space-based imaging in both the field and cluster
environments to identify a sample of ambiguous blends and observe these galaxies
with a medium resolution spectrograph. Since resolution in the redshift
dimension is all that is needed for such a study, a simple slit-based
spectrograph (with the slit properly positioned), and potentially a fiber-based
(assuming the fiber encompasses both/all galaxies), will be sufficient and with
multiplexing will be more efficient than an AO IFU.
