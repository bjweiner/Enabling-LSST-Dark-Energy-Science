\section{Weak lensing}
\label{sec:wl}

{\it B) Weak lensing: Intrinsic alignment studies, exquisite-seeing data
for morphology templates, etc. Draft notes from Rachel, to be reviewed
by Will, iterated on, and condensed before the rest of the group takes
them too seriously.}

\subsection{Galaxy morphologies}

All existing methods of inferring weak lensing shear depend in some way on knowledge of (at minimum)
the galaxy intrinsic ellipticity distribution, and in many cases they also depend on the
distribution of higher-order moments of galaxies as well.  Do we require additional observations
from US OIR facilities to obtain that information for LSST?  I (RM) had raised this question on the
telecon, but here are my current thoughts now that I consider the question in more detail.  I think
the answer depends on the method of shear inference:

\begin{itemize}
\item Regular old ``measure shapes and average to infer ensemble shear'' methods: these typically
  have implicit assumptions about the distributions of galaxy morphologies and ellipticities, which
  should be calibrated out using simulated data that has a realistic distribution of all those
  things.  Existing efforts to do this rely on HST data.  I would argue that we need more HST data
  to avoid being cosmic variance dominated for the very faintest galaxies (currently only observed
  in UDF), but that is outside the scope of this committee's efforts.  Alternatively, I think that
  the best-seeing subset of LSST data may provide some information on this, and it's unlikely that
  observations with some other telecope could provide a comparable volume of high-quality data.
\item Bayesian Fourier approaches (Bernstein \& Armstrong): this approach requires a deep (high
  $S/N$) dataset to place priors on the distributions of galaxy moments.  Their proposal is to use a
  deep subset of the same dataset used for the science; in the LSST context, this would mean using
  the DDFs.  No external data needed.
\item Hierarchical inference (Schneider et al.): A prior on the intrinsic ellipticity distribution
  has to come from somewhere.  Presumably the source could be the same as for (1), not requiring
  additional US OIR facilities outside of LSST.  It's not clear how information about detailed
  morphologies would go into this approach, but perhaps at the stage of testing calibration using
  simulations.  In this case the needs are similar to those of (1) as well.
\item Tulley-Fischer (Huff et al.~2013): This method more than the others would
  benefit from US OIR capabilities beyond LSST and space-based observatories
  (e.g., HST, WFIRST). This method requires slit or IFU spectroscopy of galaxies
  in order to determine their rotations and place them on the Tulley-Fischer
  relation which can greatly reduce the intrinsic ellipticity uncertainty (a
  dominant source of weak lensing noise). I (wd) haven't kept up with this
  method and I am not sure if it has been developed since 2013, last I heard
  they had been awarded Keck DEIMOS time to test the method. This method has
  been presented as competitive with LSST but I am not sure how complementary
  it is.
\end{itemize}

Tentative conclusion: our report doesn't have to say anything about calibrations of galaxy
morphology and ellipticity distributions, because we will get what we need either from LSST data
itself or from HST. I (wd) am also leaning towards this conclusion, however I would
like to do a little more homework on the latest status of the Tulley-Fischer
method and potential complementarity with LSST.

\subsection{Intrinsic alignments}

Intrinsic alignments (IA) of galaxy shapes with the cosmic web are a contaminant to weak lensing
measurements, since WL measurements assume that all coherent alignments are due to lensing. IA have
been robustly detected out to hundred Mpc scales, making them a serious theoretical uncertainty for
WL cosmology.  ({\em Should probably elaborate slightly about density-shape and shape-shape
  correlations, and how the former cannot be removed by just omitting nearby pairs.  But I am lazy
  and will fill this in later.}) Several methods have been developed for mitigating this systematic,
including nulling (which loses a lot of cosmological information and leads to stringent constraints
on photo-$z$ errors), forward modeling (which uses galaxy clustering, galaxy-shear, and shear-shear
correlations measurements, and involves marginalizing over a model for how the alignments enter
those measurements), and self-calibration (which does not require an {\em a priori} alignments
model, but has certain assumptions that have not yet been validated in realistic measurements).

Current data, primarily from the SDSS, provide us a template for how intrinsic alignments scale with
galaxy type, luminosity, and redshift for $z\lesssim 0.5$.  However, we need this information for
galaxies at higher redshifts, and we need better constraints on the alignments of blue galaxies, in
order to place reasonable priors when carrying out the WL analysis with LSST. Generally, this
requires both good imaging and redshift information, in order to better localize the galaxy pairs in
3D.

Question: why not use LSST itself?  Will the lSST data itself (despite photo-$z$ errors) ultimately
be the most informative dataset about IA, and perhaps we don't need strong external priors?  I think
the answer to this is that since IA are degenerate with other things (photo-$z$ errors and certain
shear systematics) it is still valuable to have {\em external} priors so this doesn't become a
limiting uncertainty for LSST WL.

Question: what type of data do we need for this?  In principle, we need a spectroscopic dataset that
covers decent-sized contiguous areas (so as to constrain alignments to tens of Mpc) but also has a
decent sampling rate within these fields (to beat down shape noise).  This seems somewhat orthogonal
to a ``many small fields to beat down cosmic variance'' approach that one might want for
spectroscopic observations to constrain photo-$z$ errors, but perhaps a compromise could be
reached.  {\em We need to make a strawman for area, sampling rate, and selection - what kinds of
  galaxies we care about.}

Question: can we use cross-correlations?  Perhaps we could use a more sparse sampling of the density
field over a wider area, and constrain IA using cross-correlations with the spectroscopic galaxies
instead of the auto-correlations of them (using a method like that from Blazek et al 2012 or Chisari
et al 2014).  {\em How to define the needed dataset for this?  Have to think about it.}
