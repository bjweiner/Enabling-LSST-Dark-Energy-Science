% ======================================================================

\section{Strong Lensing}
\label{sec:sl}

{\em C) Strong lensing: Monitoring, spectroscopy, positions (incl. IFU
spectroscopy, monitoring of lens solution for supernovae, high
resolution imaging follow-up with ELTs, spectroscopy to enable combined
SL/WL analysis of clusters).}

% For each science area, we should:
%
% 1) Describe the science need in a few sentences at most
%
% 2) Describe the needed capability (or capabilities) in enough detail
% that someone could determine whether they overlap with other needs.
% (E.g.:  “medium-resolution (R~4000-6000) spectroscopy covering the
% full optical window for i<25.3 objects with multiplexing of ~1000 over
% ~10 arcminute diameter fields” could work, or for a differentcase
% “optical medium-resolution spectroscopy with a highly-multiplexed
% spectrograph with a many-degree FoV on a 4m telescope” would also give
% enough information to allow identification of common needs).

% ----------------------------------------------------------------------

\subsection{Time Delay and Compound Lens Cosmography: High Precision Galaxy Mass Models from High Resolution Imaging and Spectroscopy}
{\it Phil Marshall, Adam Bolton and others}

The primary route to cosmology from strong lensing is time delays in
galaxy-scale lensed quasars and supernovae. Galaxy scale compound lenses
(i.e. systems with two sources at different redshifts) have also been
suggested.
To be useful as probes of cosmological distances, galaxy scale lenses
need very well constrained mass models. These constraints will come from
a) high resolution imaging of the Einstein rings due to the source
AGN or SN host galaxy, and b) spatially resolved spectroscopy of the lens
galaxy, enabling measurement of the stellar velocity dispersion field.

% We now describe the needed capability (or capabilities) in enough detail
% that someone could determine whether they overlap with other needs.

We expect to be able to compile samples of several hundred lensed AGN
and  lensed SN systems with accurately measured LSST time delays (CITE
Liao et al 2015). Snapshot imaging with either JWST or GSMTs can provide
the Einstein ring constraints needed to turn each of these systems into
a  5\% precision distance (CITE Meng et al 2015). This may not be
sufficient to enable the mass distributions to be modeled with the
required accuracy: deeper IFU data, also from JWST or GSMTs, that
provide spectroscopic constraints on the lens galaxy kinematics, are
also likely to be needed.  Similar data would be needed for the compound
lenses: here, we also  aspire to a sample of $\sim 100$ systems. While
these targeted observations would be narrow field, they would enable
some considerable ancillary science, notably  in the areas of dark
matter substructure (from perturbations to the imaged rings) and AGN
host galaxy structure. Source redshifts may  need to come from targeted
observations with 10-m class telescopes  before going to the larger
facilities: this is itself a large program.


% ----------------------------------------------------------------------

\subsection{Time Delay Cosmography: Additional Monitoring to Improve Time Delay Accuracy}
{\it Eric Linder, Phil Marshall and others}

If LSST's cadence is insufficient to provide Stage IV accurate
measurements of lens time delays, we may need to supplement the light
curves with additional monitoring data. Discuss.

We now describe the needed capability (or capabilities) in enough detail
that someone could determine whether they overlap with other needs.

\ldots

% ----------------------------------------------------------------------

\subsection{Time Delay Cosmography: Multi-object Spectroscopy to Improve the Fidelity of our Lens Environment and Line of Sight Characterization}
{\it Curtis McCully, Adam Bolton, Phil Marshall and others}

LSST will provide photometric redshifts and stellar masses of all
galaxies in the fields of each cosmographic lens. This may not be enough
to allow us to characterize the lens environments and line of sight mass
distribution with sufficient accuracy for Stage IV cosmology. Discuss.

We now describe the needed capability (or capabilities) in enough detail
that someone could determine whether they overlap with other needs.

\ldots

% ----------------------------------------------------------------------

\subsection{Cluster Mass Function: Spectroscopic Surveying in Cluster Fields}
\label{sec:sl:clusters}
{\it Will Dawson and others}

Discuss.

We now describe the needed capability (or capabilities) in enough detail
that someone could determine whether they overlap with other needs.

\ldots

% ======================================================================
